\title{A geometric framework for outlier detection in high-dimensional data}


\def\correspondingauthor{\footnote{Corresponding author, e-mail: \href{mailto:moritz.herrmann@stat.uni-muenchen.de}{moritz.herrmann@stat.uni-muenchen.de}, Department of Statistics, Ludwig Maximilians University Munich, Ludwigstr. 33, D-80539, Munich, Germany.}}
%\author[1]{Moritz Herrmann\correspondingauthor}
% %\orcidlink{0000-0002-4893-5812}}
%\author[1]{Florian Pfisterer}
% %\orcidlink{}}
%\author[1]{Fabian Scheipl}
% %\orcidlink{0000-0002-2729-0947}}
\author{Moritz Herrmann\correspondingauthor, Florian Pfisterer, and Fabian Scheipl \\
Department of Statistics, Ludwig Maximilians University, Munich, Germany}
%\affil[1]{Department of Statistics, Ludwig Maximilians University, Munich, Germany}


\vspace{-1em}
  \date{}
\begingroup
\let\center\flushleft
\let\endcenter\endflushleft
\maketitle
\endgroup
\selectlanguage{english}
%\maketitle

\textbf{Article Type:} Focus Article\\

\begin{abstract}
Outlier or anomaly detection is an important task in data analysis. We discuss the problem from a geometrical perspective and provide a framework which exploits the metric structure of a data set. Our approach rests on the \textit{manifold assumption}, i.e., that the observed, nominally high-dimensional data lie on a much lower dimensional manifold and that this intrinsic structure can be inferred with manifold learning methods. We show that exploiting this structure significantly improves the detection of outlying observations in high dimensional data. We also suggest a novel, mathematically precise and widely applicable distinction between \textit{distributional} and \textit{structural} outliers based on the geometry and topology of of the data manifold that clarifies conceptual ambiguities prevalent throughout the literature.
Our experiments focus on functional data as one class of structured high-dimensional data, but the framework we propose is completely general and we include image and graph data applications. Our results show that the outlier structure of high-dimensional and non-tabular data can be detected and visualized using manifold learning methods and quantified using standard outlier scoring methods applied to the manifold embedding vectors.
%\keywords{Anomaly detection, dimension reduction, manifold learning, functional data}
\end{abstract}
